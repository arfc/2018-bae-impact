\section{Transition Scenario Analysis}
With the two different \gls{UNF} inventories with
the same mass but different composition, we perform
transition scenarios of the U.S. nuclear fleet from
a \gls{LWR} fleet to a \gls{SFR} fleet. We ran the
transition scenario starting in 2020 with the calculated
\gls{UNF} inventory and the current U.S. nuclear fleet.
To meet the increasing energy demand, more \glspl{LWR}
are deployed until the \glspl{SFR} become available.
We varied the nuclear energy demand growth rate and
\glspl{SFR} availability, as listed on table \ref{tab:param}.


\begin{table}[h]
    \centering
    \begin{tabular}{cc}
        \hline
        Parameters & Values \\
        \hline
        Energy Demand Growth Rate [\% per year] & 0, 0.5, 1, 1.5 \\
        \gls{SFR} available year & 2030, 2035, 2040, 2045, 2050\\
        Pre-2020 \gls{UNF} inventory & Precise \gls{UDB}, Recipe \gls{UDB}, None \\
        \hline
    \end{tabular}
    \caption{Parameters used for transition scenario}
    \label{tab:param}
\end{table}


\subsection{Results - Transition Scenario Analysis}
As expected, there was no observable difference between the
precise \gls{UDB} and the recipe \gls{UDB} case. As mentioned
above, the two cases differ very little in fissile content

